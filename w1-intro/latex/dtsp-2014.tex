

%\documentclass[mathserif,9pt]{beamer}
%\documentclass[9pt]{beamer}
\documentclass[mathserif,9pt,handout]{beamer}


\usepackage{amsmath}
\usepackage{amssymb}
%\usepackage{antpolt}
\usepackage{times}
\usepackage{subfigure}
\usepackage{algorithmic}
\usepackage{gregmath}
\usepackage{graphicx}
\usepackage{url}
\usepackage{framed}
\usepackage{tikz,pgfplots}
\usepackage{tikz}
\usepackage[T1]{fontenc}
%\usetheme{Warsaw}
\usetheme{Madrid}
\usecolortheme{seahorse}


\setbeamersize{text margin left=7mm, text margin right=7mm} 
\graphicspath{{./pdf/}}
\def\mubf{\boldsymbol{\mu}}
\def\d{\mathrm{d}}
\definecolor{drexblue}{rgb}{0.12890625,0.2734375,0.48046875}
\definecolor{pantoneblue}{RGB}{41,5,161}
\def\pb{\color{pantoneblue}}

%\setbeamercolor{frametitle}{bg=pantoneblue}
%\setbeamercolor{subsection in sidebar}{fg=pantoneblue}
%\setbeamercolor{block title}{bg=pantoneblue}
%\setbeamercolor{titlelike}{bg=pantoneblue}
%\setbeamercolor{structure}{bg=black, fg=pantoneblue}
%\setbeamercolor{item}{fg=pantoneblue}
%

\usepackage{hyperref}



\begin{document}

\title[\url{gregory.ditzler@gmail.com}]{\bf Fundamentals of Deterministic Digital Signal Processing}
\author[Deterministic Digital Signal Processing]{Gregory Ditzler}
\institute[]{\scriptsize 
  Drexel University \\
  Dept. of Electrical \& Computer Engineering \\ 
  Philadelphia, PA, USA\\
  {\color{blue!50!black}\url{gregory.ditzler@gmail.com}} \\
  %{\color{blue}\url{http://gregoryditzler.com}} 
}
\date[\today]{\scriptsize \today}


%\maketitle


\begin{frame}
  \titlepage\vfill
  \vspace{-2em}
  \begin{center}
    \includegraphics[scale=.3,keepaspectratio]{drexel-logo.pdf} 
  \end{center}
\end{frame}


%--------------------------------------------------------------------------------------------
\begin{frame}\frametitle{Overview}\small
  \begin{block}{Review of Digital Signal Processing Concepts}
  \begin{itemize}
    \item {\em Clerical work before we get started}
    \item {\em Signals \& Systems}: LTI, sampling, discrete signals, quantization
    \item {\em Transforms}: $Z$-transform, discrete-time Fourier transform, fast Fourier transform 
    \item {\em Frequency Response \& Filters}: transfer functions, FIR, IIR, filter design 
  \end{itemize}
  \end{block}
  
  \begin{exampleblock}{Random Signals}
  \begin{itemize}
    \item {\em Probability \& Statistics Review}: random variables, mean, expectations, variance 
    \item {\em Random Processes}: Bernoulli
  \end{itemize}
  \end{exampleblock}
  
  \begin{alertblock}{Examples \& Homework}
  \begin{itemize}
    \item We are going to do several examples, some of which do not have their solutions in the slides, and there are homework problems that are due in two weeks. 
  \end{itemize}
  \end{alertblock}


\end{frame}



%--------------------------------------------------------------------------------------------
\begin{frame}\frametitle{Logistics}\small
  \begin{columns}
    \column{.5\textwidth}
      \begin{center}\includegraphics[width=.8\textwidth]{dtsp.jpg}\end{center}
    \column{.5\textwidth}
    {\bf\color{blue!50!black}ECES631} \\
    Fund. of Deterministic DSP \\
    \vspace{1em}
    
    {\bf\color{blue!50!black}Instructor} \\
    Dr. Gail Rosen (\url{gailr@ece.drexel.edu}) \\
    \vspace{1em}
    
    {\bf\color{blue!50!black}Office Hours} \\
    By appointment. \\
    \vspace{1em}

    
    {\bf\color{blue!50!black}Text} \\
    Oppenheim, Schafer \& Buck, ``Discrete-Time Signal Processing,'' 3rd Ed.
    \vspace{1em}
    
    {\bf\color{blue!50!black}Other Stuff}
    \begin{itemize}
      \item DSP is a {\em prerequisite}!
      \item Course materials are available on BBLearn.  
    \end{itemize}
    \vspace{1em}
  \end{columns}
\end{frame}



%--------------------------------------------------------------------------------------------
\begin{frame}\frametitle{What is DSP?}\small
  \begin{center}
    \includegraphics[width=.9\textwidth]{dsp_flow.pdf}
  \end{center}
  \begin{exampleblock}{\small Digital Signal Processing}
  \begin{itemize}
    \item {\bf\color{green!50!black}Digital} 
      \begin{itemize}
        \item Method to represent a quantity, a phenomenon or an event
        \item Why Digital?
      \end{itemize}
    \item {\bf\color{green!50!black}Signal}
      \begin{itemize}
        \item What is a signal?
        \item What are we interested in?
      \end{itemize}
    \item {\bf\color{green!50!black}Processing}
      \begin{itemize}
        \item What kind of processing do we need to perform?
        \item What special effects do we need to look out for?
      \end{itemize}
  \end{itemize}
  \end{exampleblock}
\end{frame}


%--------------------------------------------------------------------------------------------
\begin{frame}\frametitle{What is DSP?}\small
  \begin{center}
    \includegraphics[width=.9\textwidth]{dsp_flow_sampled.pdf}
  \end{center}
  \begin{block}{\small Digital Signal Processing}
  \begin{itemize}
    \item What is a digital signal? Its just a sequence of numbers that can be represented as 
      \begin{align}
         x = \{x[n]\}, \hspace{3em} -\infty < n <  \infty \nonumber
      \end{align}
    \item $x[n]$ is sampled from an analog signal
      \begin{align}
         x[n] = x(nT_s) \hspace{3em} -\infty < n <  \infty \nonumber
      \end{align}
      where $T_s$ is the sampling period, which is the reciprocal of the sampling rate ($f_s$). 
  \end{itemize}
  \end{block}
\end{frame}


%--------------------------------------------------------------------------------------------
\begin{frame}\frametitle{Common sequences and operations}\small
  {\bf\color{blue!50!black}Unit and Impulse Sequences} \\
  The discrete unit step ($u[n]$), and impulse sequences $\delta[n]$ are among the most commonly utilized sequences in DSP. Why? 
  \begin{align}
     \delta[n] = \left\{ 
         \begin{array}{l l}
           1 & \textrm{if } n=0 \\
           0 & \textrm{otherwise}
         \end{array}
       \right.
     \hspace{3em}
     u[n] = \left\{ 
         \begin{array}{l l}
           1 & \textrm{if } n \geq 0 \\
           0 & \textrm{otherwise}
         \end{array}
       \right. = \sum_{k=-\infty}^{\infty} \delta[n]
     \nonumber
  \end{align}
  
  {\bf\color{blue!50!black}Exponential Sequences} \\
  The exponential sequences is important for representing and analyzing linear time-invariant discrete-time systems
  \begin{align}
    x[n] = A \alpha^n \nonumber
  \end{align}
  where if $A,\alpha \in \Rbb$ then $x[n] \in \Rbb$.\\
  \vspace{1em} 
  
  {\bf\color{blue!50!black}Euler's Identities} \\
  Never forget!
  \begin{align}
    \cos(\omega n) = \frac{\e^{j\omega n} + \e^{-j\omega n}}{2}, \hspace{1em}
    \sin(\omega n) = \frac{\e^{j\omega n} - \e^{-j\omega n}}{j2} \nonumber \\
    \e^{j\omega n} = \cos(\omega n) + j \sin(\omega n), \hspace{1em}
    \e^{-j\omega n} = \cos(\omega n) - j \sin(\omega n)
    \nonumber
  \end{align}
\end{frame}

%--------------------------------------------------------------------------------------------
\begin{frame}\frametitle{What do they look like?}\small
  \begin{center}
    \includegraphics[width=.35\textwidth]{step_fcn.pdf} \hspace{1em}
    \includegraphics[width=.35\textwidth]{impulse_fcn.pdf}  \\
    \vspace{1em}
    
    \includegraphics[width=.35\textwidth]{exp_fcn.pdf} \hspace{1em}
    \includegraphics[width=.35\textwidth]{cos_fcn.pdf}
  \end{center}
\end{frame}


%--------------------------------------------------------------------------------------------
\begin{frame}\frametitle{Linear Systems}\small
  \begin{block}{What is a linear system?}
    A class {\em of linear systems} is defined by the property of superposition. Let $T$ be an operation, and $y_1[n]$ and $y_2[n]$ be the system responses of a system when $x_1[n]$ and $x_2[n]$ are the inputs, respectively. Then the system is linear if, and only if: 
    \begin{align}
      T\{x_1[n] + x_2[n]\} = T\{x_1[n]\} + T\{x_2[n]\} = y_1[n] + y_2[n]  \hspace{1em} \textrm{(additivity)}
      \nonumber
    \end{align}
    and 
    \begin{align}
      T\{ \alpha x[n]\} = \alpha T\{  x[n]\} = \alpha y[n] \hspace{1em} \textrm{(homogenity)}
      \nonumber
    \end{align}
    where $\alpha$ is an arbitrary constant. 
  \end{block}
  
  \begin{exampleblock}{Questions}
    
  \end{exampleblock}
\end{frame}

%--------------------------------------------------------------------------------------------
\begin{frame}\frametitle{Time-Invariant Systems}\small
  \begin{block}{What is time-invariance?}
    A system is said to be time-invariant if a delay on the input sequence results in an equal delay of the output sequence. That is, if $\hat{x}[n] = x[n-n_0]$ then $\hat{y}[n] = y[n-n_0]$. 
  \end{block}
  
  \begin{exampleblock}{The Accumulator as a Time-Invariant System}
    Define $x_i[n] = x[n-n_0]$ and let
    \begin{align}
    y[n - n_0] = \sum_{k=-\infty}^{n-n_0}x[k] \nonumber
    \end{align}
    Next, we have 
    \begin{align}
    y_1[n] = \sum_{k=-\infty}^{n}x_1[k] = \sum_{k=-\infty}^{n}x[k-n_0]\nonumber
    \end{align}
    Substituting the change of variables for $k_1=k-n_0$ into the sum gives
    \begin{align}
      y_1[n] = \sum_{k_1=-\infty}^{n-n_0}x[k_1] = y[n-n_0] 
      \nonumber
    \end{align}
  \end{exampleblock}
\end{frame}


%--------------------------------------------------------------------------------------------
\begin{frame}\frametitle{Other Important Stuff}\small
  {\bf\color{blue!50!black}Causality} \\
  A system is casual if, for every choice of $n_0$, the output sequence value at the index $n=n_0$ depends only of the input sequence values for $n \leq n_0$. Is $y[n] = x[n+1] - x[n]$ causal? How about $y[n] = \log(x[|n| - n_0])$?\\
  \vspace{1em}
  
  {\bf\color{blue!50!black}Stability} \\
  A system is bounded-input bounded-output (BIBO) stable if and only if  every bounded input sequence results in a bounded output sequence. 
  \vspace{1em}
  
  {\bf\color{blue!50!black}Everything else} \\
  Seriously, read Chapter 2 of the text book!
\end{frame}


%--------------------------------------------------------------------------------------------
\begin{frame}\frametitle{Linear Time-Invariant Systems}\small
  \begin{block}{What are they?}
    \begin{itemize}
      \item As the name states, these systems are linear and time-invariant. They are one of the most important components to the field of digital signal processing. 
      \item An LTI system can be completely characterized by its impulse response. That is, $x[n] = \delta[n]$.
    \end{itemize}
  \end{block}
  
  \begin{exampleblock}{Convolution}
    \begin{itemize}
      \item The convolution of two sequences $x$ and $h$ is defined by:
        \begin{align}
          y[n] = \sum_{k=-\infty}^{\infty} x[k] h[n-k]  = x[n] \star h[n] \nonumber
        \end{align}
      \item The importance of the equation shown above cannot be overstated enough 
    \end{itemize}
  \end{exampleblock}
\end{frame}


%--------------------------------------------------------------------------------------------
\begin{frame}\frametitle{Convolution Example}\small
  \begin{center}
    \includegraphics[width=.45\textwidth]{conv_00.pdf} \hspace{2em}
    \includegraphics[width=.45\textwidth]{conv_01.pdf}
  \end{center}
\end{frame}

%--------------------------------------------------------------------------------------------
\begin{frame}\frametitle{Convolution Example}\small
  \begin{center}
    \includegraphics[width=.45\textwidth]{conv_02.pdf} \hspace{2em}
    \includegraphics[width=.45\textwidth]{conv_03.pdf}
  \end{center}
\end{frame}


%--------------------------------------------------------------------------------------------
\begin{frame}\frametitle{Equivelence of LTI Systems}\small
  \begin{center}
    \includegraphics[height=.45\textheight]{lti_sys_00.pdf} \hspace{2em}
    \includegraphics[height=.45\textheight]{lti_sys_01.pdf}
  \end{center}
\end{frame}



%--------------------------------------------------------------------------------------------
\begin{frame}\frametitle{LTI Examples: Moving Average}\small
  \begin{exampleblock}{}
  \begin{align}
    y[n] = \frac{1}{L} \sum_{k=0}^{L-1}x[n-k]
    \nonumber
  \end{align}
  \end{exampleblock}
  \uncover<2->{
  \noindent{\bf\color{blue!50!black}Linear?: }\uncover<3->{ Yes}\\
  \begin{align}
     \frac{1}{L} \sum_{k=0}^{L-1}(a x_1[n-k] + b x_2[n-k]) = a\left(\frac{1}{L} \sum_{k=0}^{L-1}x[n-k]\right) + b\left(\frac{1}{L} \sum_{k=0}^{L-1}x[n-k] \right)
    \nonumber
  \end{align}
  }
  \uncover<4->{
  \noindent{\bf\color{blue!50!black}Time-Invariant?:} \uncover<5->{Yes}\\
  \begin{align}
    \frac{1}{L} \sum_{k=0}^{L-1}x[n-k-n_0] = \frac{1}{L} \sum_{k=0}^{L-1}x[(n-n_0)-k] = y[n-n_0]
    \nonumber
  \end{align}
  }
  \uncover<6->{
  \noindent{\bf\color{blue!50!black}Casual?:} \uncover<7->{Yes}\\
  \begin{align}
    y[n] = \frac{1}{L}\left( x[n] + x[n-1] +\ldots + x[n-L+1] \right)
    \nonumber
  \end{align}
  }
  \uncover<8->{
  \noindent{\bf\color{blue!50!black}BIBO?:} \uncover<9->{Yes}\\
  \begin{align}
    |y[n]| = \left|\frac{1}{L} \sum_{k=0}^{L-1}x[n-k]\right| \leq \frac{1}{K} \sum_{k=0}^{L-1}|x[n-k]| \leq B_x
    \nonumber
  \end{align}
  }
\end{frame}


%--------------------------------------------------------------------------------------------
\begin{frame}\frametitle{LTI Examples: Downsampler}\small
  \begin{exampleblock}{}
  \begin{align}
    y[n] = x[Mn]
    \nonumber
  \end{align}
  \end{exampleblock}
  \uncover<2->{
  \noindent{\bf\color{blue!50!black}Linear?: } \uncover<3->{Yes}\\
  \begin{align}
     a x_1[Mn] + b x_2[Mn] = a y_1[n] + b y_2[n]
    \nonumber
  \end{align}
  }
  \uncover<4->{
  \noindent{\bf\color{blue!50!black}Time-Invariant?:} \uncover<5->{No}\\
  \begin{align}
    y_1[n] = x_1[Mn] = x[Mn - n_0] \neq y[n-n_0] = x[M(n-n_0)]
    \nonumber
  \end{align}
  }
  \uncover<6->{
  \noindent{\bf\color{blue!50!black}Casual?:} \uncover<7->{No}\\
  \begin{align}
    y[-1] = x[M],\textrm{but }y[1] = x[M] 
    \nonumber
  \end{align}
  }
  \uncover<8->{
  \noindent{\bf\color{blue!50!black}BIBO?:} \uncover<9->{Yes}\\
  \begin{align}
    |y[n]| = |x[Mn]| \leq B_x
    \nonumber
  \end{align}
  }
\end{frame}


%--------------------------------------------------------------------------------------------
\begin{frame}\frametitle{Sampling}\small
  \begin{figure}\centering
    \subfigure[$x(t)$]{\includegraphics[height=.35\textheight]{sampling_p0.pdf}}
    \subfigure[sampled $x(t)$]{\includegraphics[height=.35\textheight]{sampling_p1.pdf}} \\
    \subfigure[quantized signal]{\includegraphics[height=.35\textheight]{sampling_p2.pdf}}
    \subfigure[$\hat{x}(t)$]{\includegraphics[height=.35\textheight]{sampling_p3.pdf}}  
  \end{figure}
\end{frame}

%--------------------------------------------------------------------------------------------
\begin{frame}\frametitle{Quantization}\small
  \begin{columns}
  \column{.4\textwidth}
    \begin{figure}
    \centering
    \includegraphics[height=.35\textheight]{sampling_p0.pdf}\\
    $\downarrow$ \\
    \includegraphics[height=.35\textheight]{sampling_p2.pdf}
    \end{figure}
  \column{.6\textwidth}
    %\includegraphics[width=\textwidth]{Quantization_error.png} \\
    \includegraphics[width=\textwidth]{q_err.pdf} \\
    %{\tiny\url{http://en.wikipedia.org/wiki/Quantization_(signal_processing)}}
    \vspace{1em}
    
    {\color{blue!50!black}\bf Quantization}
    \begin{itemize}
      \item Truncating a continuous signal's discrete representation to a finite set of values. 
        \begin{itemize}
          \item for example, a signal $x(t) \in (\pm 5V)$ is quantized with a resolution of $\frac{10V}{128}$
          \item a form of compression
        \end{itemize} 
      \item Quantization can be  non-uniform for the range of $x(t)$. For example, sampling of voiced speech. 
      \item Quantizing coefficients can have adverse effects to a systems response. Can you think of an example?
    \end{itemize}
  \end{columns}
\end{frame}

%--------------------------------------------------------------------------------------------
\begin{frame}\frametitle{Quantization \& the Elliptic Filter}\small
  %\begin{align}
  %  H(\omega) = \frac{1}{\sqrt{1+\epsilon^2R_n^2(\xi, \omega/\omega_0)}}
  %\end{align}
  \begin{center}
    Pole-Zero plots of an elliptic filter before and after have the coefficients $a_k$ and 
    $b_k$ quantized. \\
    \vspace{1em}
    
    \includegraphics[width=.45\textwidth]{ellip_pz_stable.pdf} \hspace{1em}
    \includegraphics[width=.45\textwidth]{ellip_pz_unstable.pdf}
  \end{center}
\end{frame}


%--------------------------------------------------------------------------------------------
\begin{frame}\frametitle{Quantization \& the Elliptic Filter}\small
  \begin{center}
    Frequency and phase response of an elliptic filter before and after have the coefficients $a_k$ and 
    $b_k$ quantized. \\
    \vspace{1em}
    
    \includegraphics[height=.55\textheight]{ellip_freq.pdf} \hspace{1em}
    \includegraphics[height=.55\textheight]{ellip_phase.pdf}
  \end{center}
\end{frame}





%--------------------------------------------------------------------------------------------
\begin{frame}\frametitle{Linear-Time Invariance}\small
\end{frame}


%--------------------------------------------------------------------------------------------
\begin{frame}\frametitle{$Z$-transform}\small
\end{frame}

%--------------------------------------------------------------------------------------------
\begin{frame}\frametitle{$Z$-transform (Example)}\small

  \begin{columns}
    \column{.5\textwidth}
      \begin{block}{Question}
      Given the system
      \begin{align}
        H(z) = \e^{z^{-1}} \nonumber
      \end{align}
      What is $h[n]$? What is the ROC? Is the system BIBO stable?
      \end{block}
    \column{.5\textwidth} \footnotesize
      \uncover<2->{
      {\bf\color{blue!50!black}Solution}\\
      Directly apply the Lorentz series expansion to $H(z)$
      \begin{align}
        H(z) &= \e^{z^{-1}} = 1 + z^{-1} + \frac{1}{2!}z^{-2} +\cdots \nonumber \\
        &= \sum_{n=0}^{\infty} h[n] u[n]\nonumber
      \end{align}
      Thus, $h[n] = \frac{1}{n!}u[n]$. Is the system BIBO stable?} \uncover<3->{Apply the ratio test to see if the series converges, diverges, or neither.
      \begin{align}
        \lim_{n\rightarrow\infty} \frac{h[n+1]}{h[n]} &= \lim_{n\rightarrow\infty} \frac{n!}{(n+1)!} = \lim_{n\rightarrow\infty} \frac{n!}{(n+1)n!}\nonumber \\
        &= \lim_{n\rightarrow\infty} \frac{1}{n+1} = 0\nonumber
      \end{align}
      Therefore, $H(z)$ is BIBO stable. } \uncover<4->{Examining $H(z)$, we see there are no poles or zeros. Only an essential singularity. ROC is all $z$ except @ 0.}
  \end{columns}
\end{frame}


%--------------------------------------------------------------------------------------------
\begin{frame}\frametitle{Discrete-Time Fourier Transform}\small
\end{frame}

%--------------------------------------------------------------------------------------------
\begin{frame}\frametitle{Parseval's Theorem}\small
  \begin{block}{\small General Idea}
    \begin{itemize} 
      \item Parseval's theorem provides a convenient way to compute the energy of a signal in the time-domain or frequency domain. In physics, this theorem is commonly referred to as the the Plancherel theorem. 
    \end{itemize}
  \end{block}
  
  \uncover<2->{
  \begin{exampleblock}{\small Derivation}
    \footnotesize
    \begin{align}
      \alert<8->{\frac{1}{2\pi} \int\limits_{-\pi}^{\pi} | X(\e^{j\omega}) |^2 \d\omega} 
         &= \frac{1}{2\pi} \int\limits_{-\pi}^{\pi}  X(\e^{j\omega}) X^*(\e^{j\omega})\d\omega 
          \uncover<3->{= \frac{1}{2\pi} \int\limits_{-\pi}^{\pi} \left( \sum_{n=-\infty}^{\infty} x[n] \e^{-j\omega n} \right) \left( \sum_{n'=-\infty}^{\infty} x^*[n'] \e^{j\omega n'} \right) \d\omega\nonumber} \\
         \uncover<4->{&= \frac{1}{2\pi} \int\limits_{-\pi}^{\pi}  \sum_{n=-\infty}^{\infty} x[n] \sum_{n'=-\infty}^{\infty} x^*[n'] \e^{j\omega (n' -n)} \d\omega \nonumber} \\
         \uncover<5->{&= \sum_{n=-\infty}^{\infty} x[n] \sum_{n'=-\infty}^{\infty} x^*[n']  \underbrace{\frac{1}{2\pi} \int\limits_{-\pi}^{\pi} \e^{j\omega (n' -n)}}_{ \left\{ \begin{tabular}{l l} $1$ & $n = n'$ \\ $0$ & \textrm{otherwise}  \end{tabular} \right. } \d\omega }
         \uncover<6->{= \sum_{n=-\infty}^{\infty} x[n]x^*[n] }
         \alert<8->{\uncover<7->{= \sum_{n=-\infty}^{\infty}| x[n]|^2 }}
         \nonumber
    \end{align}
  \end{exampleblock}
  }
\end{frame}


%--------------------------------------------------------------------------------------------
\begin{frame}\frametitle{Fast Fourier Transform}\small
\end{frame}





\end{document}
